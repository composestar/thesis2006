\section{Features Specific to \Compose*{}}
\label{section:FSTC}
The Composition Filters approach uses a restricted (pattern matching) language to define filters. This language makes it possible to reason about the semantics of the concern. 
\Compose* offers three features that use this possibility, which originate in more control and correctness over an application under construction. These features are:
\begin{description}[style=nextline,noitemsep]
\item [Ordering of filter modules] It is possible to specify how the superimposition of filter modules should be ordered. Ordering constraints can be specified in a fixed, conditional, or partial manner. A fixed ordering can be calculated exactly, whereas a conditional ordering is dependent on the result of filter execution and therefore evaluated at runtime. When there are multiple valid orderings of filtermodules on a join point, partial ordering constraints can be applied to reduce this number. These constraints can be declared in the concern definition;
\item [Filter consistency checking] When superimposition is applied, \Compose* is able to detect if the ordering and conjunction of filters creates a conflict. 
For example, imagine a set of filters where the first filter only evaluates method \emph{m} and another filter only evaluates methods \emph{a} and \emph{b}. 
In this case the latter filter is only reached with method \emph{m}; this is consequently rejected and as a result the superimposition may never be executed. There are different scenarios that lead to these kinds of problems, \eg conditions that exclude each other;
\item [Reason about semantic problems] When multiple pieces of advice are added to the same join point, \Compose* can reason about problems that may occur.
An example of such a conflict is the situation where a real-time filter is followed by a wait filter. Because the wait filter can wait indefinitely, the real-time property imposed by the real-time filter may be violated.
\end{description}
The above mentioned conflict analyzers all work on the assumption that the behavior of every filter is well-defined. This is not the case for the meta filter, its user-undefined, and therefore unpredictable, behavior poses a problem to the analysis tools. 

Furthermore, \Compose* is extended with features that enhance the usability. These features are briefly described below:
\begin{description}[style=nextline,noitemsep]
\item [Integrated Development Environment support] The \Compose* implementations all have a IDE plug-in; \Compose*[.NET] for Visual Studio, \Compose*[J] and \Compose*[C] for Eclipse;
\item [Debugging support] The debugger shows the flow of messages through the filters. It is possible to place breakpoints to view the state of the filters; 
\item [Incremental building process] When a project is build and not all the modules are changed, incremental building saves time.
\end{description}

Some language properties of \Compose* can also be seen as features, being:
\begin{description}[style=nextline,noitemsep]
\item [Language independent concerns] A \Compose* concern can be used for all the \Compose* platforms, because the composition filters approach is language independent;
\item [Reusable concerns] The concerns are easy to reuse, through the dynamic filter modules and the selector language;
\item [Expressive selector language] Program elements of an implementation language can be used to select a set of objects to superimpose on;
\item [Support for annotations] Using the selector, annotations can be woven at program elements. At the moment annotations can be used for superimposition. 
\end{description}
