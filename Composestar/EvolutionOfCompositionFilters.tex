\section{Evolution of Composition Filters}
\Compose* is the result of many years of research and experimentation.
The following time line gives an overview of what has been done in the years before and during the \Compose* project.
\hyphenation{Small-talk}
\begin{description}[noitemsep,style=nextline,leftmargin=15mm]
\item[1985] The first version of Sina is developed by Mehmet Ak\c{s}it.
            This version of Sina contains a preliminary version of the composition filters concept called semantic networks.
            The semantic network construction serves as an extension to objects, such as classes, messages, or instances.
            These objects can be configured to form other objects, such as classes, from which instances can be created.
            The object manager takes care of synchronization and message processing of an object.
            The semantic network construction can express key concepts like delegation, reflection, and synchronization~\cite{koopmans:sina95}.
\item[1987] Together with Anand Tripathi of the University of Minnesota the Sina language is further developed.
            The semantic network approach is replaced by declarative specifications and the interface predicate construct is added.
\item[1991] The interface predicates are replaced by the dispatch filter, and the wait filter manages the synchronization functions of the object manager.
            Message reflection and real-time specifications are handled by the meta filter and the real-time filter~\cite{bergmans:phd94}.
\item[1995] The Sina language with Composition Filters is implemented using Smalltalk~\cite{koopmans:sina95}.
            The implementation supports most of the filter types.
            In the same year, a preprocessor providing C++ with support for Composition Filters is implemented~\cite{glandrup:ms95}.
\item[1999] The composition filters language ComposeJ~\cite{wichman:ms99} is developed and implemented.
            The implementation consists of a preprocessor capable of translating composition filter specifications into the Java language.
\item[2001] ConcernJ is implemented as part of a \MSc thesis~\cite{salinas:ms01}.
            ConcernJ adds the notion of superimposition to Composition Filters.
            This allows for reuse of the filter modules and to facilitate crosscutting concerns.
\item[2003] The start of the \Compose* project, the project is described in further detail in this chapter.
\item[2004] The first release of \Compose*, based on \dotNET.
\item[2005] The start of the Java port of \Compose*.
\item[2006] Porting \Compose* to C is started.
\item[2006] Start of the StarLight project. This project is described in detail in \autoref{sec:ComposestarStarLight}.
\item[2007] Merged the StarLight branch with \Compose*.
\end{description}